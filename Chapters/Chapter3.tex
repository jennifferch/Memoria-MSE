\chapter{Desarrollo e implementación} % Main chapter title
\label{Chapter3} % Change X to a consecutive number; for referencing this chapter elsewhere, use \ref{ChapterX}

\definecolor{mygreen}{rgb}{0,0.6,0}
\definecolor{mygray}{rgb}{0.5,0.5,0.5}
\definecolor{mymauve}{rgb}{0.58,0,0.82}

%%%%%%%%%%%%%%%%%%%%%%%%%%%%%%%%%%%%%%%%%%%%%%%%%%%%%%%%%%%%%%%%%%%%%%%%%%%%%
% parámetros para configurar el formato del código en los entornos lstlisting
%%%%%%%%%%%%%%%%%%%%%%%%%%%%%%%%%%%%%%%%%%%%%%%%%%%%%%%%%%%%%%%%%%%%%%%%%%%%%
\lstset{ %
  backgroundcolor=\color{white},   % choose the background color; you must add \usepackage{color} or \usepackage{xcolor}
  basicstyle=\footnotesize,        % the size of the fonts that are used for the code
  breakatwhitespace=false,         % sets if automatic breaks should only happen at whitespace
  breaklines=true,                 % sets automatic line breaking
  captionpos=b,                    % sets the caption-position to bottom
  commentstyle=\color{mygreen},    % comment style
  deletekeywords={...},            % if you want to delete keywords from the given language
  %escapeinside={\%*}{*)},          % if you want to add LaTeX within your code
  %extendedchars=true,              % lets you use non-ASCII characters; for 8-bits encodings only, does not work with UTF-8
  %frame=single,	                % adds a frame around the code
  keepspaces=true,                 % keeps spaces in text, useful for keeping indentation of code (possibly needs columns=flexible)
  keywordstyle=\color{blue},       % keyword style
  language=[ANSI]C,                % the language of the code
  %otherkeywords={*,...},           % if you want to add more keywords to the set
  numbers=left,                    % where to put the line-numbers; possible values are (none, left, right)
  numbersep=5pt,                   % how far the line-numbers are from the code
  numberstyle=\tiny\color{mygray}, % the style that is used for the line-numbers
  rulecolor=\color{black},         % if not set, the frame-color may be changed on line-breaks within not-black text (e.g. comments (green here))
  showspaces=false,                % show spaces everywhere adding particular underscores; it overrides 'showstringspaces'
  showstringspaces=false,          % underline spaces within strings only
  showtabs=false,                  % show tabs within strings adding particular underscores
  stepnumber=1,                    % the step between two line-numbers. If it's 1, each line will be numbered
  stringstyle=\color{mymauve},     % string literal style
  tabsize=2,	                   % sets default tabsize to 2 spaces
  title=\lstname,                  % show the filename of files included with \lstinputlisting; also try caption instead of title
  morecomment=[s]{/*}{*/}
}

%----------------------------------------------------------------------------------------
% Resumen de capitulo
%----------------------------------------------------------------------------------------

En este capítulo se detallan las herramientas utilizadas para el desarrollo del presente trabajo. Luego se exponen los cambios fundamentales realizados sobre la estructura del código exitente, para portarlo a la EDU-CIAA-NXP y las mejoras introducidas. Se documenta además, el desarrollo de los nuevos componentes de hardware virtuales, los ports de las bibliotecas y ejemplos de programa, así como nuevos programas.

%----------------------------------------------------------------------------------------
\section{Herramientas de desarrollo}
\label{sec:Herramientas de trabajo}
%----------------------------------------------------------------------------------------

Se exponen las herramientas que se utilizan en el desarrollo de este trabajo:

\begin{itemize}
	\item EDU-CIAA-NXP: esta es la plataforma de hardware objetivo a emular del presente trabajo. Es uno de los diseños de hardware del Proyecto CIAA. En particular, el enfoque es ayudar a las Universidades Argentinas a migrar de microcontroladores de 8 bits a modernos microcontroladores de 32 bits al usar una placa diseñada en Argentina con hardware y software abiertos. Estas placas se difundieron en todas las universidades Argentinas con carreras de electrónica y afines, y también, en algunos países limítrofes. 
Se utilizó la placa física para ensayos de comparación entre lo real y el emulador web desarrollado.

	\item Visual Studio Code \citep{VisualStudioCode}: es un editor de código fuente gratuito y de código abierto desarrollado por Microsoft. Incluye soporte para la depuración, control integrado de Git, resaltado de sintaxis, finalización inteligente de código, fragmentos, refactorización de código y muchas otras herramientas más. Se eligio este IDE, de la sigla en inglés Integrated Development Environment \citep{IDE}, por la capacidad de sus herramientas y la simpleza de su editor de código. El uso de este editor facilitó la escritura y el mantenimiento del código del emulador, mejorando la productividad y la calidad del desarrollo.
	
	\item Inkscape \citep{inkscape}: es un editor de gráficos vectoriales que permite crear, editar y modificar gráficos. En el proceso de desarrollo de la interfaz del emulador web se utilizó para diseñar algunos elementos gráficos dentro del dibujo de la placa EDU-CIAA-NXP.

	\item GitHub \citep{GitHub}: es una plataforma de desarrollo colaborativo que permite alojar proyectos utilizando el sistema de control de versiones Git. Se utilizó los servicios de esta plataforma para almacenar y
compartir el código fuente, de manera que se pueda hacer un seguimiento
de las últimas modificaciones realizadas.
	
	\item Travis CI \citep{TravisCI}: es un servicio de integración continua en la nube y es utilizado mayormente para configurar y ejecutar pruebas automatizadas en un entorno controlado y reproducible. Además, se integra con sistemas de control de versiones como GitHub y permite que con cada cambio realizado en el repositorio se ejecuten las pruebas definidas en el script de construcción.
De esta manera, se asegura la calidad del software. Incluso, proporciona informes detallados de las pruebas realizadas y servicio de notificaciones por correo electrónico.

	\item GitLab \citep{GitLab}: es una plataforma web de gestión de repositorios y permite la colaboración en el desarrollo de software. Proporciona un conjunto completo de herramientas para el ciclo de vida del desarrollo de software, por lo tanto, permite configurar pipelines de integración y entrega continua, en consecuencia, automatiza la compilación, las pruebas y el despliegue de software de manera eficiente. Es una alternativa a otras plataformas como GitHub con Travis CI.
	
	\item DigitalOcean \citep{DigitalOcean}: ofrece servicios de infraestructura de computación en la nube, tales como permitir a los usuarios crear y administrar servidores virtuales, conocidos como Droplets. Incluso, ofrece opciones de almacenamiento, configuración de redes privadas virtuales, servicios de bases de datos, entre otros. DigitalOcean se destaca por su enfoque en la simplicidad y la facilidad de uso de su plataforma. El emulador desarrollado en el presente trabajo se encuentra desplegado en el servidor de DigitalOcean.

\end{itemize}

%----------------------------------------------------------------------------------------
\section{Introducción al desarrollo}
%----------------------------------------------------------------------------------------

Luego de comprender el funcionamiento de \textit{Arm Mbed OS Simulator}, se comenzó el desarrollo del emulador eliminando módulos y dependencias innecesarias, tales como: mbed-http, simple-mbed-cloud-client, features, rtos y todas las sub-carpetas dentro de la carpeta \textquotedbl target\textquotedbl{}  excluyendo la sub-carpeta TARGET\_SIMULATOR. 
Se modificaron las carpetas y archivos de configuración para adaptarlos al desarrollo del entorno del emulador para la placa EDU-CIAA-NXP. Luego, se reemplazó \textquotedbl mbed-os\textquotedbl{}  por la biblioteca \textit{\textbf{sAPI}}, pero se mantuvieron algunos componentes propios de Mbed, como \textquotedbl events\textquotedbl{}  y \textquotedbl callback\textquotedbl, para agilizar el desarrollo.

La figura \ref{fig:estructuraCiaa} exhibe la nueva estructura de carpetas y archivos, realizada para el \textit{Emulador de la placa EDU-CIAA-NXP}, el cual a partir de aquí se nombra como \textit{EmuCIAA}.

\begin{figure}[ht]
	\centering
	\includegraphics[scale=.57]{./Figures/estructuraCiaa.jpg}
	\caption{Estructura de carpetas y archivos de \textit{EmuCIAA}.}
	\label{fig:estructuraCiaa}
\end{figure}




%----------------------------------------------------------------------------------------
\section{Cambios realizados sobre la estructura del código exitente}
\label{sec:Cambios realizados sobre la estructura del código exitente}
%----------------------------------------------------------------------------------------

%-----------------------------------------------
\subsection{Nuevas tecnologías agrgadas}

Cambios en el Frontend

\begin{itemize}
    \item SVG EDU-CIAA-NXP \citep{SVGFirmwareV3}, SVG GLCD 128x64, SVG LCD DISPLAY 20x4, SVG DHT11, SVG Potenciómetro \(10K\Omega\), SVG Termistor NTC, SVG Analog Stick, SVG LEDs: esta reutilización fue necesaria para brindar a los usuarios una experiencia visual interactiva. Asimismo, se incorporaron fuentes tipográficas en los dibujos SVG para las representaciones de texto. Estas fuentes  están definidas en información vectorial,  lo que permitió una visualización nítida y escalable en diferentes tamaños para las pantallas LCD.

     \item Mocha \citep{Mocha}: es un marco de trabajo para pruebas de JavaScript que tiene funciones que se ejecutan en NodeJS y en el navegador web. En consecuencia, hace que las pruebas asincrónicas sean simples. Asimismo, Las pruebas se ejecutan en serie y se realiza el envío de excepciones aún no detectadas a los casos de prueba correctos.
El uso de este marco de trabajo permitió hacer pruebas sobre la interfaz de usuario de la plataforma.
     
     \item Chai \citep{Chai}: es una biblioteca de aserciones que puede usarse con cualquier marco de pruebas de Javascript. Asimismo, tiene varias interfaces: \textit{assert}, \textit{expect} y \textit{should}, que permiten al desarrollador elegir cuál usar. Chai se utiliza en las pruebas del emulador web para  verificar el comportamiento esperado de las funciones, componentes y datos generados por el emulador. 
\end{itemize}

Cambios en el Backend

\begin{itemize}
    \item Check \citep{Check}: es una biblioteca de pruebas unitarias para el lenguaje de programación C que proporciona un conjunto de macros y funciones que facilitan la escritura y la ejecución de las pruebas unitarias. Además, provee mecanismos que aislan y ejecutan las pruebas en un entorno controlado y separado, usando suites de pruebas, funciones de inicialización y limpieza. Se utilizó en el emulador web para verificar el correcto funcionamiento de las funciones y componentes implementados en el backend del emulador escritos en lenguaje C.
    
    \item CMocka \citep{CMocka}: es una biblioteca de pruebas unitarias especialmente diseñada para C, destacandose por su capacidad de crear mocks (falsos) y stubs (simulaciones) de funciones. De esta manera se logra probar componentes de código que dependen de funciones externas. Y, además, facilita el aislamiento de las unidades de código y la creación de escenarios de prueba que pueden ser controlados. El uso de esta tecnología permitió simular funciones mediante mocks para controlar el comportamiento de las funciones dependientes y facilitar las pruebas de código que interactúa con dichas funciones.
    
    \item sAPI CIAA \citep{sAPICIAA}: esta biblioteca escrita en lenguaje C y compatible con C++ implementa una API simple que funciona como una capa de abstracción de hardware para microcontroladores. Es la principal biblioteca del Proyecto CIAA para el desarrollo de aplicaciones en C/C++ en los frameworks Firmware v2\citep{firmwareV2} y Firmware v3\citep{firmwareV3}. 
Para la emulación a nivel de API, se utilizó como base la API de la sAPI v0.6.2 disponible en firmware v3 del Proyecto CIAA \citep{sAPIv0.6.2} y se realizarón las implementaciones necesarias para que funcione en la web, en lugar de funcionar en el hardware de un microcontrolador real.
De esta manera, se proporcionó una interfaz idéntica, permitiendo a los usuarios del emulador, programar en la plataforma web de la misma manera que lo harían con la placa EDU-CIAA-NXP real, logrando que cualquier programa escrito utilizando la sAPI pueda correr en el emulador.
Cabe destacar que al emular a nivel de sAPI, el usuario no podrá utilizar funciones de bajo nivel de la EDU-CIAA-NXP, como ser la biblioteca del fabricante del microcontrolador (LPCopen\citep{lpcopen}), o el acceso directo a registros físicos del microcontrolador, al igual que sucede con el emulador \textit{Arm Mbed OS Simulator}.
\end{itemize}


%----------------------------------------------------------------------------------------
\section{Frontend}
%----------------------------------------------------------------------------------------

En esta parte de la plataforma web se implementó el desarrollo de la interacción entre el \textit{backend} y el navegador del usuario.

%------------------------------------------------------
\subsection{Diseño de la Interfaz de Usuario}

Para el desarrollo de la interfaz, se optó por un diseño intuitivo, de manera que el usuario se sienta familiarizado con las herramientas de trabajo y que la disposición de los componentes sea cómoda y esté organizada al momento de usarlas.

Para su estudio, la figura \ref{fig:PlataformaEmulador1} muestra la plataforma web del Emulador de la placa EDU-CIAA-NXP, dividida en tres partes.

\begin{figure}[ht]
	\centering
	\includegraphics[scale=.28]{./Figures/PlataformaEmulador.png}
	\caption{Plataforma de emulación para la placa EDU-CIAA-NXP.}
	\label{fig:PlataformaEmulador1}
\end{figure}

Por consiguiente, el diseño de la interfaz de usuario de la plataforma proporciona las siguientes áreas:

\begin{itemize}
	\item Área de ensamblado: el valor predeterminado muestra la placa EDU-CIAA-NXP, y también se permite agregar componentes.
	\item Área de codificación: se proporciona un editor de código en línea para programar con la placa EDU-CIAA-NXP. La primera vez que se accede a la platafoma se muestra en ejecución un ejemplo de código predeterminado.
	\item Área de consola integrada: se muestra en una ventana la salida que se vería a través del puerto serie. 
\end{itemize}

Debido a las áreas de trabajo que presenta la plataforma, el usuario programador podrá realizar las siguientes tareas:

\begin{itemize}
	\item Ver los programas de ejemplo predeterminados.
	\item Crear un nuevo proyecto.
	\item Ejecutar un programa de ejemplo o uno nuevo.
	\item Editar programas.
	\item Visualizar los cambios programados para la placa virtual.
	\item Agregar nuevos componentes.
	\item Ver los errores obtenidos en la programación.
	\item Ver lo programado en la salida de consola.
\end{itemize}

%--------------------------------------
\subsubsection{Área de ensamblado}

Para el desarrollo de la placa EDU-CIAA-NXP y componentes externos se usaron dibujos en formato de gráficos vectoriales bidimensionales (SVG) por las siguientes características:

\begin{itemize}
	\item Son más ligeras, entonces se cargan más rápido en el navegador.
    \item Por su capacidad de ser modificado por medio de \textit{JavaScript}. Por lo tanto, se pudieron crear imágenes interactivas.
	\item Evitan que las imagenes se deformen y no pierden calidad.
	\item Permite programar animaciones.
\end{itemize}

En ese sentido, para la capa de programación \textit{JavaScript UI}, se pudo implementar el comportamiento interactivo para los botones de la placa (TEC1, TEC2, TEC3 y TEC4) usando el código SVG generado para la placa EDU-CIAA-NXP, incluso, se pudo modificar también el comportamiento de los LEDs, que permitió mostrar dinamicamente en la placa el encendido y apagado.

Además, con el objetivo de optimizar la experiencia del usuario, esta área permite elegir uno o más dispositivos virtuales de entrada/salida desde la barra lateral \textit{sidebars} derecho. En esta barra se exhiben todos los periféricos disponibles para la plataforma web. 

Una vez que el usuario elige un periférico de la barra lateral y realiza la configuración de las conexiones, la barra automáticamente colapsa, ocultándose del área de ensamblado. Al colapsarse, se muestra el periférico integrado en la aplicación. Asimismo,  el usuario tiene la opción de volver a expandir la barra lateral para agregar un nuevo periférico.

También, al añadir un periférico virtual, el usuario tiene la posibilidad de eliminarlo. Para realizarlo, simplemente debe seleccionar el periférico deseado, lo que habilitará el icono de \textquotedbl Eliminar periféricos externos\textquotedbl. Al hacer clic en el icono, el periférico virtual es eliminado de forma automática y efectiva.

En la figura \ref{fig:AgregarPeriferico} se observa que el usuario puede elegir qué componente agregar a la aplicación y en la figura \ref{fig:AgregarPeriferico2} se muestra que el componente se agregó a la aplicación.

\begin{figure}[ht]
	\centering
	\includegraphics[scale=.37]{./Figures/AgregarPeriferico.png}
	\caption{El usuario puede elegir un componente en la aplicación.}
	\label{fig:AgregarPeriferico}
\end{figure}

\begin{figure}[ht]
	\centering
	\includegraphics[scale=.37]{./Figures/AgregarPeriferico2.png}
	\caption{Periférico agregado en el área de ensamblado. }
	\label{fig:AgregarPeriferico2}
\end{figure}

%-----------------------------------------
\subsubsection{Área de codificación}

Esta parte de la plataforma se reserva al usuario para que pueda programar sus propias aplicaciones. Esta ventana de edición presenta las siguientes capacidades:

\begin{itemize}
\item Manejar la sintaxis para el lenguaje C.
\item Soportar el uso de las constantes, como por ejemplo: '\#define'.
\item Permitir el uso de las palabras claves, comentarios, etc.
\item Permitir el resaltado de líneas de código, sangría automática y número de línea.
\item Utilizar la función buscar (ctrl + f).
\item Utilizar la función buscar/reemplazar (ctrl + h).
\item Utilizar la función rehacer (ctrl + y).
\end{itemize}

Para compilar un programa, la plataforma provee al usuario el botón “ejecutar”. Sin embargo, si hubo problemas de sintaxis, errores lógicos, etc., se mostrarán esos errores al usuario.

En la figura \ref{fig:PlataformaErrores2} se muestra el código que generó los errores de compilación y en la figura \ref{fig:PlataformaErrores1} se observa los errores de compilación en el área de ensamblado.

\begin{figure}[ht]
	\centering
	\includegraphics[scale=.42]{./Figures/PlataformaErrores1.png}
	\caption{Código que generó los errores de compilación.}
	\label{fig:PlataformaErrores2}
\end{figure}

\begin{figure}[ht]
	\centering
	\includegraphics[scale=.39]{./Figures/PlataformaErrores2.png}
	\caption{Errores de compilación.}
	\label{fig:PlataformaErrores1}
\end{figure}

También, se implementó una estructura jerárquica en la lista desplegable de ejemplos. El propósito es organizar y presentar los ejemplos agrupados por periféricos, de manera más ordenada y fácil de navegar para el usuario.

La figura \ref{fig:listExamples} muestra la estructura jerárquica de la lista de ejemplos.

\begin{figure}[ht]
	\centering
	\includegraphics[scale=.42]{./Figures/listExamples.jpg}
	\caption{Estructura jerárquica de ejemplos. }
	\label{fig:listExamples}
\end{figure}

Ademas, al seleccionar algún ejemplo que contiene un periférico externo de la lista desplegable, la aplicación, automáticamente carga dentro del área de ensamblado el periférico con las conexiones a los pines configurados por defecto. Es decir, el usuario no tiene la necesidad de seleccionar y configurar el periférico desde la barra lateral del área de ensamblado.

La figura \ref{fig:cargarPeriferico} muestra el periférico agregado automáticamente al seleccionar el ejemplo \textquotedbl potentiometer\textquotedbl.

\begin{figure}[ht]
	\centering
	\includegraphics[scale=.21]{./Figures/cargarPeriferico.png}
	\caption{Carga automática del periférico. }
	\label{fig:cargarPeriferico}
\end{figure}

%------------------------------------------
\subsubsection{Área de consola integrada}

Dentro del código traducido en el proceso de compilación por \textit{Emscripten}, se encuentran las siguientes funciones definidas y configuradas previamente, las cuales son: 

\begin{itemize}
\item \texttt{print}, esta función envía el texto a la terminal de la interfaz gráfica del emulador web, al utilizar la función \texttt{terminal.write}.
\item \texttt{printErr}, se comunica con la consola de error del navegador, al usar \newline \texttt{console.error}.
\end{itemize}

Ambas funciones interactuan con el código \textit{JavaScript}. La función \texttt{printErr} se comunica con la consola de error del navegador y la función \texttt{print} se comunica con la terminal del emulador web a través de la biblioteca xterm.js, que es un componente de terminal de front-end escrito en \textit{JavaScript} y que permite construir terminales en el navegador. 

Entre las principales características de xterm.js se destaca:

\begin{itemize}
	\item Funciona con la mayoría de las aplicaciones de terminal, como bash, es compatible con aplicaciones basadas y eventos de mouse.
	\item Es de alto rendimiento, por eso es realmente rápido.
	\item No requiere de dependencias externas para funcionar. La dependencia principal para el funcionamiento básico es el propio navegador web.
	\item API bien documentada.
\end{itemize}

Además de las características destacadas, xterm.js es una biblioteca que fue adoptada por diversos proyectos populares, tales como VS Code, Hyper y Theia. La amplia adopción de xterm.js por parte de estos proyectos contribuyo a la expansión de su comunidad de desarrolladores, quienes brindan un sólido respaldo y soporte.

En la figura \ref{fig:Terminal2} se puede observar la salida por consola de un programa y en la figura \ref{fig:Terminal1} se muestra el programa de usuario que generó la salida por consola.

\begin{figure}[ht]
	\centering
	\includegraphics[scale=.59]{./Figures/Terminal2.png}
	\caption{Salida de la terminal serie.}
	\label{fig:Terminal2}
\end{figure}

\begin{figure}[ht]
	\centering
	\includegraphics[scale=.42]{./Figures/Terminal1.png}
	\caption{Programa de usuario que imprime por consola.}
	\label{fig:Terminal1}
\end{figure}

%---------------------------------
\subsection{JavaScript UI}

Esta capa se desarrolló con el propósito de que se comunique con la capa \textit{JavaScript HAL}, y también con el objetivo de proporcionar componentes de código, utilidades y manejo de eventos en la aplicación de usuario.

Por tanto, para lograr la comunicación con la capa \textit{JavaScript HAL}, se desarrolló en esta capa los objetos que se sucriben al detector de eventos programados en la \textit{HAL}.

\textit{JavaScript} permite crear oyentes, utilizando el método \texttt{on()} y pasando como argumento el nombre del evento al que se quiere suscribir. De esta manera, se  programaron varios subscriptores para un mismo evento en diferentes archivos \textit{JavaScript} dentro de esta capa. En consecuencia, se logró una mayor interactividad entre los componentes de la plataforma.

Es decir, cuando se emite algún evento en la \textit{HAL}, entonces el oyente suscrito a ese evento en esta capa \textit{UI} lo podrá escuchar y realizar las acciones que correspondan para la funcionalidad requerida. La figura \ref{fig:EventemitterNodeJSUI} describe esta situación.

\begin{figure}[ht]
	\centering
	\includegraphics[scale=.52]{./Figures/EventemitterNodeJSUI.png}
	\caption{Diagrama de bloques de los oyentes de \textit{EventEmitter} en la capa UI.}
	\label{fig:EventemitterNodeJSUI}
\end{figure}

%---------------------------------------
\subsection{Aplicación de Usuario}

La plataforma de emulación para la placa EDU-CIAA-NXP es una aplicación en línea, que se ejecuta en el \textit{browser} del usuario. La interfaz fue diseñada como una herramienta que permite al usuario realizar sus tareas de programación dentro de una plataforma simple e intuitiva.

Sin embargo, presenta una limitación en las unidades de tiempo especificadas por los valores en las constantes, por ejemplo cuando el usuario define \texttt{TASK1\_PERIODICITY} con un valor de 1000 para ser usado dentro del codigo siguiente: 
	
\begin{lstlisting}[caption={Ejemplo TASK1\_PERIODICITY}]
#define TASK1_PERIODICITY 1000} 

if( task1Counter++ == TASK1_PERIODICITY ){
  task1();
  task1Counter = 0;
}
\end{lstlisting}	

Significa que la tarea  planificada se ejecutará cada 1000 milisegundos en la placa fisica. Sin embargo, en el emulador web, la velocidad de ejecución puede variar y no necesariamente coincidir con el tiempo real. Por lo tanto, el tiempo de ejecución de cada iteración de la tarea \texttt{task1} podría ser más lento en el emulador web. Estas diferencias en la ejecución se deben a las limitaciones inherentes de \textit{JavaScript} y \textit{Emscripten}, que pueden afectar la precisión del tiempo.

%-------------------------
\section{Backend}

En esta capa de programación se desarrolló toda la lógica necesaria para emular las funcionalidades que proporcionan las bibliotecas: \textit{\textbf{sAPI}}, \textit{\textbf{freeRTOS}} y \textit{\textbf{seos\textunderscore pont}} para la placa EDU-CIAA-NXP.

%-----------------------------
\subsection{Biblioteca C}

En primer lugar, se identificaron las funciones de las bibliotecas \textit{C} originales para empezar a emular. Luego, en el emulador se crearon interfaces que reflejen las estructuras de las bibliotecas originales, que incluyo definiciones de funciones, estructuras de datos y constantes.

Es decir, en las funciones originales se examinaron los parámetros de entrada y los valores de retorno, para luego mapearlos correctamente en las definiciones de las funciones del emulador.  A modo de referencia se muestra en la tabla \ref{tab:gpioMap} las definiciones de las funciones para \texttt{sapi\_gpio.h}, que incluye los nombres de la funciones, los tipos de parámetros y el tipo de valor de retorno. Cabe destacar que estas definiciones son idénticas tanto en las \textit{\textbf{sAPI}} como en el emulador, lo que permite una fácil correspondencia entre ambas.

\begin{table}[h]
	\centering
	\caption[Módulo \textit{GPIO}]{Módulo \textit{GPIO}}
	\begin{tabular}{l c c}    
		\toprule
		\textbf{Nombre de la función} 	 & \textbf{Parámetros} 		& \textbf{Tipo de retorno}  \\
		\midrule
		gpioInit & gpioMap\_t, gpioInit				&  bool\_t \\		
		gpioRead	 & gpioMap\_t				&  bool\_t \\
		gpioWrite	 & gpioMap\_t, bool\_t				&  bool\_t \\
		gpioToggle	 & gpioMap\_t				&  bool\_t \\
		\bottomrule
		\hline
	\end{tabular}
	\label{tab:gpioMap}
\end{table}

Al igual que en la biblioteca \textit{\textbf{sAPI}} del proyecto CIAA, los archivos de código fuente para la plataforma de emulación, conservan el mismo nombre, por ejemplo  \texttt{sapi\_gpio.c}. Sin embargo, la implementación de las funciones es totalmente distinta. En el caso de la biblioteca \textit{\textbf{sAPI}} del proyecto CIAA, para \texttt{sapi\_gpio.c}  se incluyen los archivos de encabezado: \texttt{gpio\_18xx\_43xx.h} y \newline \texttt{scu\_18xx\_43xx.h}. Y en el caso de la plataforma de emulación se usa otro archivos de encabezado como \texttt{gpio\_api.h} para replicar el mismo comportamiento.

A continuación, se presenta una comparación entre las clases del módulo GPIO de la biblioteca \textit{\textbf{sAPI}} del proyecto CIAA y el módulo GPIO implementado en la plataforma de emulación. 

La figura \ref{fig:GPIOsAPI} muestra las dependencias que se usan para la implementación de \newline \texttt{sapi\_gpio.c} del proyecto CIAA y en la figura \ref{fig:GPIOEmulador} se muestran las dependencias utilizadas en el emulador para el mismo módulo, logrando el \textit{port} al emulador de las funciones para el manejo del periférico GPIO.

\begin{figure}[ht]
	\centering
	\includegraphics[scale=.41]{./Figures/DiagramaClasesGPIOsAPI.png}
	\caption{Diagrama de dependencias del módulo \textit{GPIO} de la biblioteca \textit{\textbf{sAPI}} del proyecto CIAA.}
	\label{fig:GPIOsAPI}
\end{figure}

\begin{figure}[ht]
	\centering
	\includegraphics[scale=.41]{./Figures/DiagramaClasesEmulador.png}
	\caption{Diagrama de dependencias del módulo \textit{GPIO} de la plataforma de emulación para la placa EDU-CIAA-NXP.}
	\label{fig:GPIOEmulador}
\end{figure}

Asimismo, se utilizó un esquema de nomenclatura de los archivos de encabezado y de código fuente similar al de las bibliotecas originales.  Esto permitió mantener una estructura organizada y coherente en la emulación, que facilita el mantenimiento y comprensión.

También, se reutilizó el archivo de encabezado \texttt{sapi.h} que cumple con la misma funcionalidad que en la biblioteca \textit{\textbf{sAPI}}, la cual consiste en incluir todos los módulos que conforman la biblioteca para utilizarla en el programa de usuario. 

Además, se reutilizaron los archivos de encabezado: \texttt{sapi\_datatypes.h} y \newline \texttt{sapi\_peripheral\_map.h} incluidos en todos los módulos de la biblioteca \textit{\textbf{sAPI}} del proyecto CIAA. Esta reutilización busca emular las características de hardware y prevalecer el uso de todos los tipos de datos básicos y configuraciones de la placa.

En la tabla \ref{tab:ConfiguracionGPIO} se muestran los tipos de datos de \texttt{sapi\_peripheral\_map.h} que se usan en la plataforma de emulación. Se puede observar que se reutilizó los nombres: TEC1, TEC2, TEC3 y TEC4 para los botones y los nombres LEDR, LEDG, LEDB, LED1, LED2 y LED3 para los LEDs de la placa EDU-CIAA-NXP.

\begin{table}[h]
	\centering
	\caption[\texttt{sapi\_peripheral\_map.h}.]{Tipos de datos de \texttt{sapi\_peripheral\_map.h} que se reutilizan en en la plataforma de emulación.}
	\begin{tabular}{l c c c}    
		\toprule
		\textbf{P2 header} & \textbf{P1 header} & \textbf{LEDs}  & \textbf{Switches}\\
		\midrule
		GPIO8, GPIO7, GPIO5 & T\_FIL1 &  LEDR &  TEC1\\		
		GPIO3, GPIO1, LCD1 & T\_COL2  & LEDG &  TEC2\\
		LCD2, LCD3, LCDRS & T\_COL0 & LEDB &  TEC3\\
		LCD4, SPI\_MISO, ENET\_TXD1 & T\_FIL2 & LED1 & TEC4\\
		ENET\_TXD0, ENET\_MDIO, ENET\_CRS\_DV & T\_FIL3 & LED2 & \\
	    ENET\_MDC, ENET\_TXEN, ENET\_RXD1 & T\_FIL0 & LED3 & \\
	    GPIO6, GPIO4, GPIO2 & T\_COL1&  & \\
	    GPIO0, LCDEN, SPI\_MOSI, ENET\_RXD0 & CAN\_TD&  & \\
		\bottomrule
		\hline
	\end{tabular}
	\label{tab:ConfiguracionGPIO}
\end{table}

%--------------------
\subsection{C HAL}

La capa de abstracción de hardware en \textit{C (C HAL)}  permitió replicar el comportamiento del hardware de la placa, lo que a su vez posibilitó la compatibilidad de las bibliotecas de nivel superior escritas en \textit{C} en el entorno de emulación de la plataforma web.

En esta capa se encuentra la biblioteca \textit{emscripten.h}, la cual provee las funciones y macros necesarias para interactuar con el compilador de \textit{Emscripten}. El compilador transforma el código \textit{C} a \textit{JavaScript}, de esta manera, se facilita la interacción y comunicación entre el código \textit{C} y el entorno web, donde se ejecuta el código ya convertido a \textit{JavaScript}.

\textit{Emscripten} se ejecuta en el entorno de \textit{Node.js}. para compilar el código \textit{C} a \textit{JavaScript}. Además, incluye la biblioteca \textit{emscripten.h} presentes en la capa \textit{C (C HAL)}. Esto permite que el código \textit{C} use estas funciones nativas para interactuar con el entorno de \textit{JavaScript}, llamando a funciones \textit{JavaScript} desde código \textit{C} o viceversa.

El proceso de compilación con \textit{Emscripten} involucra los siguientes pasos:

\begin{itemize}
	\item Preprocesamiento: donde prepara el código fuente antes de la compilación real del código \textit{C} de manera similar a un compilador tradicional. Esto incluye el manejo de directivas como la inclusión de archivos de cabecera \texttt{\#include}, directivas que permiten la inclusión condicional de código: \newline \texttt{\#ifdef}, \texttt{\#ifndef}, \texttt{\#else}, \texttt{\#endif}, la expansión de macros \texttt{\#define}, etc. Lo que facilita el trabajo del compilador.
	
	\item Compilación: \textit{Emscripten} utiliza LLVM para compilar el código \textit{C} en un formato intermedio binario llamado  \textit{bitcode}. 
Además. \textit{LLVM} proporciona múltiples componentes que pueden intercambiarse entre sí, a diferencia de los compiladores \textit{GCC} que presentan una estructura monolítica. 

	\item Optimización: Después de obtener el \textit{bitcode}, el compilador busca mejorar el rendimiento, la eficiencia y reducir el tamaño del código resultante, por tanto,  aplica diversas técnicas de optimizaciones antes de traducirlo a \textit{JavaScript} o WebAssembly. Estas optimizaciones pueden incluir eliminación de código muerto, reordenamiento de instrucciones, detección y eliminación de código redundante, \textit{inlining de funciones}, entre otras.
	
	\item Generación de código \textit{JavaScript}: Finalmente, \textit{Emscripten} toma el \textit{bitcode} optimizado y lo traduce a código \textit{JavaScript}.

\end{itemize}

En la figura \ref{fig:Emscripten} se muestra el diagrama con el principal funcionamiento de \textit{Emscripten}. 

\begin{figure}[ht]
	\centering
	\includegraphics[scale=.47]{./Figures/Emscripten.png}
	\caption{Diagrama de funcionamiento de \textit{Emscripten}.}
	\label{fig:Emscripten}
\end{figure}

En ese sentido, para la aplicación de usuario dentro de la plataforma, cuando se ejecuta un programa de usuario, \textit{Emscripten} utiliza una función del módulo de \textit{Node.js} para la generación de un identificador único que será utilizado para los archivos generados. Este identificador esta compuesto por el prefijo \texttt{user\_} y un número entero que representa el tiempo en milisegundos. Es decir, \textit{Emscripten} creará los siguientes archivos:

\begin{itemize}
	\item Archivo \texttt{user\_tiempoenmilisegundos.js}: resultado final de la compilación y contiene el código \textit{JavaScript} que representa el programa \textit{C}.
	\item Archivo \texttt{user\_tiempoenmilisegundos.wasm}: contiene el código binario equivalente al código \textit{C} compilado. Se utiliza si el navegador del usuario admite \textit{WebAssembly} y proporciona un rendimiento mejorado en comparación con el código \textit{JavaScript} puro.
	\item Archivo \texttt{user\_tiempoenmilisegundos.js.components}: contiene datos utilizados por el programa.
	\item Archivo \texttt{user\_tiempoenmilisegundos.wasm.map}: contiene rutas de mapeos a los archivos de código \textit{C} que fueron compilados en formato \textit{Json}.
	\item Archivo \texttt{user\_tiempoenmilisegundos.wast}: representa el módulo \textit{WebAssembly} generado, pero en una representación de texto legible.
\end{itemize}

Estos archivos se ubicarán dentro del directorio de salida  \textit{outUser}, el cual fue predefinido en la configuración de la aplicación.

%-----------------------------
\subsection{JavaScript HAL}

Esta capa de programación se diseñó para proporcionar la funcionalidad de distribuir los eventos entre los componentes de la interfaz de usuario de \textit{JavaScript} y la capa \textit{C HAL}. Para lograr este objetivo se usaron las clases \textit{EventEmitter} del módulo \textit{Events} que monitorizan y activan los eventos. Además, facilita la interacción del navegador con el código \textit{JavaScript} y la actualización de la interfaz de usuario de manera flexible y eficiente.

También, la clase \textit{EventEmitter} se basa en el modelo de publicación/suscripción que se trata de un paradigma de envío de mensajes asíncrono mediante el cual un usuario publica mensajes y uno o varios objetos se suscriben a esos eventos.

En la figura \ref{fig:PublicarSuscribir} se muestra el modelo de \textit{publicación/suscripción}.

\begin{figure}[ht]
	\centering
	\includegraphics[scale=.49]{./Figures/PublicarSuscribir.png}
	\caption{Modelo de \textit{publicación/suscripción}.}
	\label{fig:PublicarSuscribir}
\end{figure}

Para la implementacion de esta capa de emulación, se crearon archivos \textit{JavaScript} con instancias de la clase \textit{EventEmitter}, que al utilizar el método \textit{emit} lanzan eventos con nombre. El nombre del evento es un string y permite que los oyentes registrados al evento sean notificados. La figura \ref{fig:EventemitterNodejs} muestra el diagrama de bloques de la instancia de \textit{EventEmitter} en la plataforma de emulación.

\begin{figure}[ht]
	\centering
	\includegraphics[scale=.49]{./Figures/EventemitterNodejs.png}
	\caption{Diagrama de bloques de \textit{EventEmitter} implementado en la plataforma.}
	\label{fig:EventemitterNodejs}
\end{figure}

En la sección de \ref{sec:caso_de_estudio}, se mostrará la implementación de este modelo en los archivos \textit{JavaScript} del emulador web.

\subsection{\textit{\textbf{sapi\_gpio}}}

Para comenzar a emular la biblioteca \textit{sapi\_gpio} del proyecto CIAA, se identificaron las funciones principales que el entorno de la plataforma web debería ofrecer al usuario. La siguiente tabla \ref{tab:sapiGPIO} muestra las funciones del archivo de código fuente \textit{sapi\_gpio} en la biblioteca \textit{\textbf{sAPI}} del proyecto CIAA que también están presentes en el emulador.

\begin{table}[h]
	\centering
	\caption[Funciones \texttt{sapi\_gpio}]{Funciones \texttt{sapi\_gpio}}
	\begin{tabular}{p{0.20\linewidth} p{0.50\linewidth}  p{0.20\linewidth}}    
		\toprule
		\textbf{Función} 	 & \textbf{Parámetros} 		& \textbf{Tipo de retorno}  \\
		\midrule
		gpioInit & gpioMap\_t pin, gpioInit\_t config 		&  bool\_t \\		
		gpioRead	 & gpioMap\_t pin			&  bool\_t \\
		gpioWrite	 & gpioMap\_t pin, bool\_t value			& bool\_t \\
		gpioToggle	 & gpioMap\_t pin				&  bool\_t \\
		\bottomrule
		\hline
	\end{tabular}
	\label{tab:sapiGPIO}
\end{table}

Además, para lograr replicar el comportamiento de cada función emulada de manera que, al invocarlas, produzcan resultados similares a los que se obtendrían al utilizar las funciones con el hardware físico, se utilizó la capa de \textit{C HAL}. Esta capa interactúa con la capa \textit{Javascript HAL}, que se encarga de comunicarse con la interfaz de usuario, permitiendo mostrar el comportamiento de los pines GPIO programados en la aplicación del usuario.

Entonces, se utilizó la capa de emulación correspondiente a \textit{C HAL} para emular las siguientes funcionalidades:

\begin{itemize}
	\item \texttt{Chip\_GPIO\_Init(LPC\_GPIO\_PORT)}se utiliza para inicializar y configurar el hardware de los pines GPIO de la placa. En la capa de emulación \textit{C HAL}, utilizando  \textit{Emscripten}, también se realizaron configuraciones de variables para representar los tipos de pines e inicializarlos.
	
	\item \texttt{Chip\_GPIO\_SetDir (LPC\_GPIO\_PORT, gpioPort, \newline (1 \<<\<<  gpioPin), GPIO\_OUTPUT)}se utiliza para establecer si un pin GPIO se utilizará para recibir datos (entrada) o enviar datos (salida). De manera similar, en la capa de abstracción de datos en \textit{C}, se implementaron configuraciones similares para la capa \textit{Javascript HAL} usando las macros de \textit{Emscripten}.
	
	\item \texttt{Chip\_GPIO\_SetPinState(LPC\_GPIO\_PORT, gpioPort, \newline gpioPin, 0)} se utiliza para gestionar el estado de los pines GPIO, permitiendo configurarlos como salidas y establecer su valor (alto o bajo). En la capa de abstracción de datos \textit{C}, esta función actualiza la estructura de datos utilizada para almacenar información sobre la configuración de los pines GPIO. Además, registra el valor del pin especificado como parámetro.

	\item \texttt{Chip\_GPIO\_ReadPortBit(LPC\_GPIO\_PORT, gpioPort, \newline gpioPin)} se utiliza para obtener el estado actual de un pin GPIO específico en la placa, lo que permite leer datos provenientes de dispositivos externos o de otros componentes conectados a los pines GPIO. En la capa de emulación \textit{C HAL}, se emula la lectura del estado de un pin GPIO utilizando la información almacenada en la estructura de datos.
\end{itemize}

Para emular las funciones mencionadas anteriormente, se utilizó la macro \newline \texttt{EM\_ASM\_}, mediante la cual se incrustó código \textit{JavaScript} directamente en el código \textit{C}. Este código \textit{JavaScript} incrustado se compila junto con el código C y se ejecutará en el entorno de ejecución de \textit{Emscripten}  cuando la aplicación de usuario invoque a esas funciones. La figura \ref{fig:gpioEmscripten} muestra el diagrama de bloques de la macro \texttt{EM\_ASM\_}.

\begin{figure}[ht]
	\centering
	\includegraphics[scale=.40]{./Figures/gpioEmscripten.png}
	\caption{Diagrama de bloques de la macro \texttt{EM\_ASM\_}.}
	\label{fig:gpioEmscripten}
\end{figure}

Asimismo, para emular las interacciones entre la interfaz de usuario y las GPIO TEC1, TEC2, TEC3 y TEC4, se utilizó en la capa \textit{C HAL} la macro \newline \texttt{EMSCRIPTEN\_KEEPALIVE}. Su funcionamiento se explica en la siguiente sección \ref{sec:sapi_tick}. 

A continuación, en la capa \textit{Javascript HAL}, se distribuyen eventos a la capa \textit{Javascript UI} para notificarle que el pin GPIO ha sido configurado desde la capa \textit{C HAL}. Estos eventos incluyen información que será útil para gestionar los pines GPIO en la capa de interfaz de usuario. La figura \ref{fig:gpioEmit} muestra el diagrama de bloques del envío de eventos de la capa \textit{Javascript HAL} a la capa \textit{Javascript UI}.

\begin{figure}[ht]
	\centering
	\includegraphics[scale=.40]{./Figures/gpioEmit.png}
	\caption{Diagrama de bloques del envío de eventos de la capa \textit{Javascript HAL}.}
	\label{fig:gpioEmit}
\end{figure}

Por consiguiente, en la capa \textit{Javascript UI} se manipulan esos eventos, y en consecuencia, se actualiza la interfaz de la plataforma web para mostrar al usuario los cambios.

%-----------------------------------------
\subsection{\textit{\textbf{sapi\_tick}}}
\label{sec:sapi_tick}

En la tabla \ref{tab:sapiTick} se puede observar las funciones del archivo de código fuente \texttt{sapi\_tick} en la biblioteca \textit{\textbf{sAPI}} del proyecto CIAA y en el emulador.

\begin{table}[h]
	\centering
	\caption[Funciones \texttt{sapi\_tick}]{Funciones \texttt{sapi\_tick}}
	\begin{tabular}{p{0.20\linewidth} p{0.50\linewidth}  p{0.20\linewidth}}    
		\toprule
		\textbf{Función} 	 & \textbf{Parámetros} 		& \textbf{Tipo de retorno}  \\
		\midrule
		tickInit & tick\_t tickRateMSvalue 		&  bool\_t \\		
		tickRead	 & void				&  tick\_t \\
		tickWrite	 & tick\_t ticks 				& void \\
		tickCallbackSet	 & callBackFuncPtr\_t tickCallback, void* tickCallbackParams				&  bool\_t \\
		tickPowerSet & bool\_t power 		&  void \\	
		\bottomrule
		\hline
	\end{tabular}
	\label{tab:sapiTick}
\end{table}

Para emular a nivel de API, se tuvo como objetivo replicar el comportamiento de la función \texttt{tickInit}, la cual se encarga de la inicialización y configuración de la interrupción del temporizador \texttt{SysTick\_Config} en la placa física. Sin embargo, al realizar la emulación en la plataforma web, esta función no se encuentra disponible de forma nativa. Por lo tanto, fue necesario emular su comportamiento y proporcionar una alternativa compatible.

Para emular la funcionalidad de \texttt{SysTick\_Config}, se utilizó la capa de emulación correspondiente a \textit{C HAL}. Esta capa de emulación permitió ejecutar código \textit{JavaScript} en el contexto de \textit{Emscripten}, lo que posibilitó replicar el comportamiento del temporizador \textit{SysTick}. 

Una vez habilitada la interrupción del temporizador, se realiza una invocación periódica a la función \texttt{tickerCallback}, que tiene la misma implementación que en \texttt{sapi\_tick} de la biblioteca \textit{\textbf{sAPI}} del proyecto CIAA. La función \newline \texttt{tickerCallback} realiza las siguientes acciones: incrementa los contadores de ticks y, si el puntero \texttt{tickHookFunction} no es nulo, ejecuta la función establecida como \textit{calback} mediante la función de \textit{\textbf{sAPI}} \texttt{tickCallbackSet()}, pasando los parámetros \texttt{callBackFuncParams}. En consecuencia, esto permite la ejecución de tareas específicas programadas por el usuario en cada interrupción del temporizador periódico.

En el capítulo 4 se detallarán las diferencias encontradas al realizar las pruebas entre la placa y el emulador utilizando estas funciones.

En el contexto de la emulación a nivel de API, para implementar las demás bibliotecas de las \textit{\textbf{sAPI}}, se siguió el mismo esquema utilizado en \texttt{sapi\_tick}. Primeramente, se identificaron las funciones que requerían interacción con el hardware de la placa. Luego, en la capa \textit{C HAL} se implementaron funciones de emulación con \textit{Emscripten} para reflejar el comportamiento del hardware.

Para emular el comportamiento de la interrupción del temporizador \textit{SysTick} y proporcionar la invocación periódica a la función \texttt{tickerCallback} de \textit{sapi\_tick}, se utilizó la macro \texttt{EMSCRIPTEN\_KEEPALIVE} de \textit{Emscripten}, que le dice al compilador de \textit{Emscripten} que conserve la función marcada con esta macro en el código compilado, incluso si no es accedida desde el código \textit{JavaScript} del lado del cliente.

Es decir, cuando la función marcada con la macro \texttt{EMSCRIPTEN\_KEEPALIVE} sea invocada desde la capa \textit{JavaScript HAL}, llamará a la función \texttt{tickerCallback} de la biblioteca \textit{C} y la ejecutará. En la figura \ref{fig:tickerCallback} se muestra el funcionamiento de \texttt{EMSCRIPTEN\_KEEPALIVE}. 

\begin{figure}[ht]
	\centering
	\includegraphics[scale=.55]{./Figures/tickerCallback.png}
	\caption{Diagrama de bloques \texttt{EMSCRIPTEN\_KEEPALIVE}.}
	\label{fig:tickerCallback}
\end{figure}

Para lograr la interacción con la capa de emulación \textit{C HAL}, y realizar la invocación periódica a la función que usa la macro \texttt{EMSCRIPTEN\_KEEPALIVE} de \textit{Emscripten} se configuró en esta capa de desarrollo un temporizador de \textit{JavaScript}.

Además, dentro del temporizador, se utilizó la función \texttt{ccall} de \textit{Emscripten}, que permite invocar a la funcion \texttt{tickerCallback} desde el código \textit{C} compilado con \textit{Emscripten}. 

A continuacion, se muestra en la figura  \ref{fig:ccall} el funcionamiento de \texttt{ccall}. 

\begin{figure}[ht]
	\centering
	\includegraphics[scale=.49]{./Figures/ccall.png}
	\caption{Diagrama de bloques de la función \textit{ccall}.}
	\label{fig:ccall}
\end{figure}

Sin embargo, debido a la naturaleza asíncrona de \textit{JavaScript} y al uso de la función \texttt{ccall}, la función no mantiene el contexto entre las ejecuciones del temporizador. En consecuencia, cada vez que se reinicia el temporizador y se ejecuta la función  \texttt{tickerCallback}, la tarea específica programada por el usuario comienza desde el principio en lugar de continuar desde el punto donde quedó anteriormente. 

%-----------------------------------------
\subsection{\textit{\textbf{sapi\_delay}}}

La tabla \ref{tab:sapiDelay} expone las funciones del archivo de código fuente \texttt{sapi\_delay} en la biblioteca \textit{\textbf{sAPI}} del proyecto CIAA y en el emulador.

\begin{table}[h]
	\centering
	\caption[Funciones \texttt{sapi\_delay}]{Funciones \texttt{sapi\_delay}}
	\begin{tabular}{p{0.30\linewidth} p{0.40\linewidth}  p{0.20\linewidth}}    
		\toprule
		\textbf{Función} 	 & \textbf{Parámetros} 		& \textbf{Tipo de retorno}  \\
		\midrule
		delayInaccurateMs & tick\_t delay\_ms 		&  void \\		
		delayInaccurateUs	 & tick\_t delay\_us			&  void \\
		delayInaccurateNs	 & tick\_t delay\_ns				& void \\
		delay	 & tick\_t duration\_ms				&  void \\
		delayInit & delay\_t * delay, tick\_t duration 		&  void \\
		delayRead & delay\_t * delay 		&  bool\_t \\
		delayWrite & delay\_t * delay, tick\_t duration 		&  void \\	
		\bottomrule
		\hline
	\end{tabular}
	\label{tab:sapiDelay}
\end{table}

La función \texttt{delay} en la biblioteca \textit{\textbf{sAPI}} del proyecto CIAA crea una pausa en la ejecución del programa durante el tiempo especificado en \texttt{duration\_ms} implementando un bucle de espera. Este bucle se ejecutará mientras la diferencia de tiempo entre \texttt{tickRead()} y startTime(inicio actual de \texttt{tickRead()} sea menor que \texttt{duration\_ms/ tickRateMS}.

Sin embargo, cuando \textit{Emscripten} compila código \textit{C} a \textit{JavaScript}, la función \texttt{delay} tal como está escrita, causa que la ejecución de la plataforma web se bloquee o congele. Esto se debe a que la función \texttt{tickRead()} no se actualiza a la velocidad que se espera, lo que lleva a que se obtenga el mismo valor repetidamente. Es decir, debido a la naturaleza asincrónica de \textit{JavaScript} y al no tener una pausa controlada en el bucle (como \texttt{delay(1)}), \textit{JavaScript} no tiene tiempo suficiente para actualizar el valor de \texttt{tickRead()} entre iteraciones. De esta manera,  el bucle \texttt{while} se queda esperando activamente e impide al navegador atender otros eventos.

Por esta razón, se decidio usar las funciones nativas de \textit{Emscripten} en la capa de emulación \textit{C HAL}. En consecuencia, se aprovecho su eficiencia y precisión.

Para emular las funciones de espera de la biblioteca \textit{C} se implemento la función \texttt{emscripten\_sleep}, que utiliza funciones asincrónicas internas de \textit{Emscripten} para realizar pausas. 

Por lo tanto, permite al navegador atender otros eventos mientras el programa se encuentra en espera. Es decir, evita el bloqueo de la ejecución del resto del código y también, que la página no responda.

Además, proporciona pausas precisas, debido a que, \textit{Emscripten}  utiliza las capacidades de temporización del navegador para garantizar que el tiempo indicado sea realizado.

La figura \ref{fig:emscriptenDelay} representa el funcionamiento de  \texttt{emscripten\_sleep}. 

\begin{figure}[ht]
	\centering
	\includegraphics[scale=.50]{./Figures/emscriptenDelay.png}
	\caption{Diagrama de bloques \texttt{emscripten\_sleep.}}
	\label{fig:emscriptenDelay}
\end{figure}

%---------------------------------------
\subsection{\textit{\textbf{freeRTOS}}}

Para emular la funcionalidad de las tareas de \textit{freeRTOS} en el contexto del emulador web, se utilizó la biblioteca de eventos de \textit{mbed}. Entonces, para las funciones \texttt{xTaskCreate} y \texttt{xTaskCreateStatic}, se programaron funciones periódicas utilizando las siguientes funciones de la biblioteca de \textit{Mbed events}: 

 \begin{itemize}
	\item \texttt{int equeue\_create}:  crea una cola de eventos, configura e inicializa los recursos de plataforma necesarios, como semáforos y mutexes.
	
	\item \texttt{int equeue\_call\_every}:  se utiliza para crear un evento periódico en la cola de eventos equeue, programando llamadas repetidas a una función en intervalos regulares.
	
	\item \texttt{int equeue\_post}: permite publicar un evento en la cola de eventos equeue, estableciendo el tiempo y estableciendo el evento en la cola para su posterior procesamiento.
	
	\item \texttt{void equeue\_dispatch}: se encarga de despachar los eventos en la cola de eventos equeue de manera continua, verificando los tiempo y realizando acciones específicas según la configuración.

	\item \texttt{void equeue\_destroy}: permite liberar y limpiar todos los recursos asociados a una cola de eventos, libera los mutexes, semáforos y memoria asignada.
\end{itemize}

Estas funciones permiten ejecutar tareas periódicas en intervalos de tiempo regulares, lo que proporcionó una aproximación simplificada a la funcionalidad de tareas en el emulador web. Aunque esta solución no ofrece todas las características de un sistema operativo de tiempo real completo como \textit{freeRTOS}, fue adecuada para emular el funcionamiento de programas de usuario simples.

Es importante destacar que la implementación de tareas en el emulador web tiene una limitación significativa. Debido a que solo puede ejecutar un subproceso (hilo de ejecución) a la vez, no es posible que se ejecuten tareas simultáneas. Esto significa que, a diferencia del sistema operativo de tiempo real \textit{freeRTOS}, donde se pueden crear múltiples tareas que se ejecutan de manera concurrente, en el emulador web solo es posible ejecutar una sola tarea. Por lo tanto, esta solución es adecuada para programas de usuario simples que no requieran multitarea.

La tabla \ref{tab:ConceptosRTOS} expone algunos de los conceptos importantes de \textit{freeRTOS} que se cumplen en el emulador.

\begin{table}[h]
\centering
\caption[Conceptos importantes de \textit{freeRTOS} que se cumplen en el emulador.]{Conceptos importantes de \textit{freeRTOS} que se cumplen en el emulador.}
\begin{tabular}{p{0.45\linewidth} p{0.15\linewidth}  p{0.15\linewidth}}
\toprule
\textbf{Capacidades} 
& \textbf{\textit{freeRTOS}}
& \textbf{Emulador}
\\
\midrule
Multitareas & Si & No  \\
Funciones de espera &  Si & Si \\
Cambio de contexto &  Si & Si \\
Tarea de procesamiento continuo &  Si & Si \\
Manejo de prioridades & Si & No  \\
\bottomrule
\hline
\end{tabular}
\label{tab:ConceptosRTOS}
\end{table}

En el emulador, se encuentran implementados varios conceptos importantes de \textit{freeRTOS}, como funciones de espera y cambio de contexto. Sin embargo, en esta primera versión del emulador, no se han incluido el manejo de multitareas y de prioridades presentes en \textit{freeRTOS}.

%------------------------------------------
\subsection{\textit{\textbf{sapi\_dht11}}}

Al emular los periféricos externos de la biblioteca \textit{\textbf{sAPI}} del proyecto CIAA, se continuó con la misma lógica de programación utilizada para interactuar con los periféricos de la placa EDU-CIAA-NXP. Asimismo, se mapearon las funcionalidades ofrecidas por la biblioteca \textit{\textbf{sAPI}} a la plataforma web. En la siguiente tabla \ref{tab:sapiDht11} se muestran las funciones presentes en ambas plataformas.

\begin{table}[h]
	\centering
	\caption[Funciones \texttt{sapi\_dht11}]{Funciones \texttt{sapi\_dht11}}
	\begin{tabular}{p{0.20\linewidth} p{0.50\linewidth}  p{0.20\linewidth}}    
		\toprule
		\textbf{Función} 	 & \textbf{Parámetros} 		& \textbf{Tipo de retorno}  \\
		\midrule
		dht11Init & int32\_t gpio		&  void \\		
		dht11Read	 & float *phum, float *ptemp	&  bool\_t \\
		\bottomrule
		\hline
	\end{tabular}
	\label{tab:sapiDht11}
\end{table}

En esta primera versión de la plataforma web, no se ofrece la capacidad gráfica de las interacciones virtuales entre la placa y los periféricos externos. Sin embargo, el usuario debe realizar las configuraciones manualmente, eligiendo las conexiones correctas, que luego serán verificadas en la capa de \textit{JavaScript UI}.
Además, en \textit{JavaScript UI}, principalmente se centró el trabajo de emular el envío de los datos de temperatura y humedad del sensor DHT11 al microcontrolador.

Entonces, en la capa de abstracción de datos \textit{C HAL} se implementó la lectura de los datos provenientes de la capa \textit{JavaScript HAL} al usar la macro \texttt{EM\_ASM\_INT} de \textit{Emscripten}. A continuación, en la figura \ref{fig:dht11Emscripten} se presenta el diagrama de bloques de la macro \texttt{EM\_ASM\_INT}.

\begin{figure}[ht]
	\centering
	\includegraphics[scale=.53]{./Figures/dht11Emscripten.png}
	\caption{Diagrama de bloques de la macro \texttt{EM\_ASM\_INT}.}
	\label{fig:dht11Emscripten}
\end{figure}

La capa \textit{JavaScript HAL} recibe los datos enviados desde \textit{JavaScript UI} y los transmite a la capa \textit{C HAL}. La generación de los datos  emulados de temperatura y humedad, se realiza en \textit{JavaScript UI} a través de dos opciones elegidas por el usuario:
 
 \begin{itemize}
	\item Obtener los datos de temperatura y humedad local conectándose a una central meteorológica a través de la geolocalización del navegador del usuario. Sin embargo, si el servidor donde se encuentra desplegada la plataforma web no puede acceder al servicio de geolocalización del navegador por motivos de seguridad, o el usuario no permite el acceso, entonces se realizará la consulta a la central meteorológica utilizando la ubicación predeterminada de la ciudad de Buenos Aires. Acto seguido se actualizará la interfaz gráfica con los datos de temperatura y humedad.
	
	\item Generar los datos manualmente haciendo click en la interfaz gráfica que representa a la temperatura y humedad. De esta manera, el usuario puede generar los datos según su elección.
\end{itemize}

En la figura \ref{fig:uiDht11} se muestra el diagrama de bloques de la capa de interfaz de usuario \textit{JavaScript UI} con las dos opciones de usuario.

\begin{figure}[ht]
	\centering
	\includegraphics[scale=.53]{./Figures/uiDht11.png}
	\caption{Diagrama de bloques de la capa \textit{JavaScript UI} con las dos opciones para el usuario.}
	\label{fig:uiDht11}
\end{figure}

%---------------------------------------------
\subsection{\textit{\textbf{sapi\_adc}}}

Para comenzar, se realizó el mapeo de las funciones del módulo de la biblioteca \textit{\textbf{sAPI}} a la plataforma web. La tabla \ref{tab:sapiADC} expone las funciones presentes en ambas plataformas.


\begin{table}[h]
	\centering
	\caption[Funciones \texttt{sapi\_adc}]{Funciones \texttt{sapi\_adc}}
	\begin{tabular}{p{0.20\linewidth} p{0.50\linewidth}  p{0.20\linewidth}}    
		\toprule
		\textbf{Función} 	 & \textbf{Parámetros} 		& \textbf{Tipo de retorno}  \\
		\midrule
		adcInit & adcInit\_t config		&  void \\		
		adcRead	 & adcMap\_t analogInput	&  uint16\_t \\
		\bottomrule
		\hline
	\end{tabular}
	\label{tab:sapiADC}
\end{table}

El módulo \textit{adc} fue implementado en el emulador  y es utilizado por varios periféricos externos, que incluyen: el potenciómetro, el termistor NTC y el joystick. De esta manera, permite al usuario realizar pruebas de funcionamiento de un \textit{adc} real. 

Luego, se implementaron las funciones de inicialización y de lectura del componente \textit{adc}  en la capa \textit{C HAL}. Posteriormente, son utilizadas por la capa \textit{JavaScript HAL} para la interacción con el hardware y permitir al usuario trabajar con los periféricos externos en un entorno web. La figura \ref{fig:adcEmscripten} presenta el diagrama de bloques de las capas: \textit{C HAL} y \textit{JavaScript HAL} para el \textit{adc}.

\begin{figure}[ht]
	\centering
	\includegraphics[scale=.53]{./Figures/adcEmscripten.png}
	\caption{Diagrama de bloques de \textit{C HAL} y \textit{JavaScript HAL} para el módulo \textit{adc}.}
	\label{fig:adcEmscripten}
\end{figure}

En la capa \textit{JavaScript UI}, se implementó la obtención de datos para los periféricos externos que interactúan con el \textit{adc}. Por ejemplo, para el potenciómetro, los datos son establecidos por el usuario a través de la interfaz gráfica utilizando el componente HTML  \textit{input} de tipo \textit{range}. De esta manera, el usuario al deslizar este componente dentro de un rango mínimo y máximo establecido, se va realizando el siguiente cálculo para obtener el valor del \textit{adc} correspondiente:

\begin{lstlisting}[caption={Cálculo del ADC para el potenciómetro.}]
  window.JSHal.gpio.write(self.dataPin.ADC, range.value/ 3.3 * 1023);
\end{lstlisting}

En consecuencia, los cálculos actualizados se muestran en la interfaz gráfica del emulador web.

En el caso del termistor NTC, se implementó en la interfaz grafica un elemento gráfico (termómetro) para representar la temperatura en grados celsius. A medida que el usuario ajusta el termómetro, la temperatura en kelvin se va actualizando en función de la temperatura en grados celsius y, además, el valor del \textit{adc} se actualiza mediante la siguiente función:

\begin{lstlisting}[caption={Cálculo ADC del termistor NTC.}]
     ThermistorNTC.prototype.updateSampleADC= function(R_NTC) {
        let R_NTC_float = parseFloat(R_NTC);
        let R_10k_float = parseFloat(R_10k);
        let Vsupply_float = parseFloat(Vsupply);    
        let VoutT = (R_NTC_float * Vsupply_float) / (R_10k_float + R_NTC_float);
        Vout = parseFloat(VoutT.toFixed(2)); 

        this.sample = parseFloat((Vout * 1023.0 / Vsupply).toPrecision(4));
        console.log('this.sample ', this.sample);
        window.JSHal.gpio.write(this.dataPin.ADC, this.sample);     
    };
\end{lstlisting}

Además, se implementó un componente web para el joystick, encargado de gestionar los movimientos y acciones en las interacciones con el usuario. Los datos de los movimientos de los ejes X e Y del joystick son obtenidos y se utilizan para realizar cálculos de voltajes en las respectivas resistencias de cada eje, así como también, para calcular el valor del \textit{adc}. A modo de referencia se muestra las cálculos que se hicieron para el eje X del joystick:

\begin{lstlisting}[caption={Cálculo de la resistencia del eje X y del ADC.}]
            var VRx = Joy.GetVRx();
            var voltage = (Math.floor(VRx/ 3.3 * 1023)* 3.3 / 1023.0).toFixed(2);
            joyVRx.textContent = voltage;
            joyADCx.textContent = Math.trunc(VRx/ 3.3 * 1023);
            window.JSHal.gpio.write(self.dataPin.ADCx, VRx/ 3.3 * 1023);
\end{lstlisting}

Luego, para cada periférico externo, el cálculo del \textit{adc} es envíado a la capa \textit{JavaScript HAL}.
En la figura \ref{fig:uiADC} se muestra el diagrama de bloques con la interaccion de ambas capas de desarrollo.

\begin{figure}[ht]
	\centering
	\includegraphics[scale=.57]{./Figures/uiADC.png}
	\caption{Diagrama de bloques con la interaccion de las capas  \textit{JavaScript HAL} y \textit{JavaScript UI}.}
	\label{fig:uiADC}
\end{figure}


%----------------------------------------------------------------------------------------
\section{Caso de estudio}
\label{sec:caso_de_estudio}
%----------------------------------------------------------------------------------------

El usuario escribe o modifica el programa llamado \texttt{blinky} en lenguaje \textit{C}, que utiliza la biblioteca \textit{\textbf{sAPI}} para interactuar con los periféricos de hardware GPIO y controlar el LED. A continuación, ejecuta el programa dentro de la plataforma web. Para lograr esto, Node.js se encarga de ejecutar los comandos necesarios para que \textit{Emscripten} realice la compilación del código \textit{C}, que incluye: 

\begin{itemize}
	\item El codigo de la aplicacion de usuario escrito en lenguaje \textit{C}.
	\item El archivo \texttt{sapi\_gpio.c} de la capa \textit{Bibioteca C}.
	\item El archivo \texttt{gpio\_api.c} de la capa \textit{C HAL}.
\end{itemize} 

El proceso de compilación comienza con el preprocesamiento del código \textit{C}, que incluye el manejo de directivas del preprocesador como \texttt{\#include} y \texttt{\#define}. Luego, el compilador utiliza \textit{LLVM} para compilar el código \textit{C} en \texttt{bitcode}. Después, de obtener el \texttt{bitcode} realiza optimizaciones para mejorar el rendimiento y reducir el tamaño del código resultante. Finalmente, \textit{Emscripten} toma el bitcode optimizado y lo traduce a código \textit{JavaScript}, lo que permite que el programa escrito originalmente en \textit{C} pueda ser ejecutado dentro del entorno web. Los archivos resultantes de este proceso incluyen: 


\begin{itemize}
	\item \texttt{user\_tiempoenmilisegundos.js}.
	\item \texttt{user\_tiempoenmilisegundos.wasm}.
	\item \texttt{user\_tiempoenmilisegundos.wast}.
	\item \texttt{user\_tiempoenmilisegundos.js.components}.
	\item \texttt{user\_tiempoenmilisegundos.wasm.map}.
\end{itemize}

Una vez que los archivos \texttt{.js} y \texttt{.wasm} se han generado a partir del código \textit{C}  mediante \textit{Emscripten}, pueden interactuar con el código \textit{JavaScript} de las capas: \textit{JavaScript HAL} y \textit{JavaScript UI}. Ahora bien, desde estos archivos \textit{JavaScript} de la \textit{HAL} y \textit{UI}, se pueden invocar directamente las funciones \textit{C} compiladas como si fueran funciones \textit{JavaScript} regulares, y podrán ser ejecutadas en el entorno del navegador. Por lo tanto, esto permite que el programa \textit{C} interactúe con el resto del código \textit{JavaScript} de la aplicación web y que las funciones \textit{C} puedan ser utilizadas y llamadas de manera transparente en el navegador.


La figura \ref{fig:DiagramaSecuencia} muestra la interacción del usuario con el sistema y el orden en que se producen. Además, se muestra los mensajes que se pasan entre las dependencias.


\begin{figure}[ht]
	\centering
	\includegraphics[scale=.49]{./Figures/DiagramaSecuencia.png}
	\caption{Interacción del usuario con las dependencias del emulador.}
	\label{fig:DiagramaSecuencia}
\end{figure}

Además, la capa \textit{JavaScript HAL}, que interactúa con el código \textit{JavaScript} resultante de la compilación de la \textit{Biblioteca C} y \textit{C HAL}, se encarga de notificar los eventos ocurridos en esas capas para el programa de usuario \texttt{blinky}. Mediante la función \texttt{write}, se realiza la activación del evento que escribe en la GPIO, Como resultado, emitirá el evento con el nombre \texttt{gpio\_write}, pasando como argumentos el número de pin, el valor digital y el tipo de pin declarado.

La figura \ref{fig:GPIOEventEmitter} muestra el diagrama en bloques del evento con el nombre \newline \texttt{gpio\_write}.


\begin{figure}[ht]
	\centering
	\includegraphics[scale=.50]{./Figures/GPIOEventEmitter.png}
	\caption{Activación de evento con el nombre \texttt{gpio\_write}.}
	\label{fig:GPIOEventEmitter}
\end{figure}



En ese sentido, en la capa \textit{JavaScript UI}  cuando se emite el evento con el nombre \texttt{gpio\_write}, cualquier oyente que esté suscrito a ese evento podrá escucharlo y realizar las acciones correspondientes para la funcionalidad que se requiere. En este caso, la acción solicitada es encender el LED.

La figura \ref{fig:ListeningGPIOEventEmitter} muestra el diagrama en bloques del oyente subscrito al evento \texttt{gpio\_write}.


\begin{figure}[ht]
	\centering
	\includegraphics[scale=.50]{./Figures/ListeningGPIOEventEmitter.png}
	\caption{GPIO oyente del evento con el nombre \texttt{gpio\_write}.}
	\label{fig:ListeningGPIOEventEmitter}
\end{figure}

Entonces, en la plataforma web se muestrarán los cambios de \texttt{gpio\_write} en la placa virtual. 

En la figura \ref{fig:AplicacionUsuarioLeds} se presenta para una función de la GPIO, la interacción entre todas las capas de programación.


\begin{figure}[ht]
	\centering
	\includegraphics[scale=.36]{./Figures/AplicacionUsuarioLeds.png}
	\caption{Interacción entre todas las capas de programación.}
	\label{fig:AplicacionUsuarioLeds}
\end{figure}


%----------------------------------------------------------------------------------------
\section{Despliegue}
%----------------------------------------------------------------------------------------

Para hacer público el emulador de la plataforma en un servidor web, se realizó el proceso de despliegue en el servidor de \textit{DigitalOcean} mediante los siguientes pasos:

\begin{itemize}
	\item Crear una cuenta: implicó registrar una cuenta en la plataforma. Después de iniciar sesión, se creo un \textit{droplet}, que es un servidor virtual. Luego, se instaló el sistema operativo \textit{Ubuntu}, y se configuró la región geográfica donde se ubicaría el servidor y la cantidad de RAM.
	
	\item Acceso al servidor: después que el \textit{droplet} fue creado, se pudo  obtener la dirección IP pública y la clave \textit{SSH} para acceder al servidor.
	
	\item Configuración: implicó instalar las herramientas, como \textit{Mbed CLI} y \textit{Emscripten}, y también, los servicios necesarios, como los entornos de ejecución de \textit{Node.js} y \textit{Python}. Además, se realizó las configuraciones de las reglas de \textit{firewall}.

	\item Copiar la plataforma web al servidor: se utilizó \textit{GitHub}  para clonar el repositorio en el servidor. El código del presente trabajo, se encuentra en el repositorio de GitHub \citep{repositorioEmulador}.
	
	\item Iniciar el emulador web: se utilizó la herramienta \textit{TMUX} para mantener la sesión y la ventana de la terminal donde se ejecuta la aplicación de forma persistente, incluso cuando se cierra la conexión  \textit{SSH}. El presente trabajo, se encuentra actualmente ejecutandose en \url{http://134.209.168.175:7900}.
\end{itemize}
