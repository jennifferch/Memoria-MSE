% Chapter Template

\chapter{Conclusiones} % Main chapter title

\label{Chapter5} % Change X to a consecutive number; for referencing this chapter elsewhere, use \ref{ChapterX}

En este capítulo se presentan los aspectos más relevantes del trabajo realizado y
se identifican los pasos a seguir.

%----------------------------------------------------------------------------------------

%----------------------------------------------------------------------------------------
%	SECTION 1
%----------------------------------------------------------------------------------------

\section{Objetivos alcanzados}

En el trabajo realizado se logró diseñar e implementar una plataforma de emulación para la placa EDU-CIAA-NXP mediante tecnología web. Se destacan a continuación los aportes del
trabajo.

\begin{itemize}

\item El desarrollo de una plataforma de emulación para la placa EDU-CIAA-NXP que realiza un aporte al proyecto CIAA y a la comunidad de sistemas
embebidos en general.

\item El diseño de un sistema modular y flexible que permite agregar fácilmente nuevas
funcionalidades.

\item El desarrollo de una plataforma abierta que permite la colaboración de otros desarrolladores.

\item La implementación de un sistema usable que facilita el aprendizaje y promueve la enseñanza de
programación en sistemas embebidos.

\item El desarrollo de una plataforma que es especialmente útil para realizar un prototipado rápido o pruebas de concepto sin depender de la placa.

\item Emulación a nivel de API de la biblioteca sAPI del proyecto CIAA.

\item Implementación de ejemplos funcionales predeterminados en la plataforma de emulación.

\item Implementación de periféricos externos virtuales de la placa EDU-CIAA-NXP en formato SVG.

\item La realización de pruebas de acceso y de funcionamiento para validar los resultados que se esperan de la plataforma on-line.

\item Implementación de pruebas unitarias y de integración en la interfaz de usuario que verifican el cumplimiento de los requisitos funcionales.

\item La creación de una herramienta que puede ser una nueva rama de desarrollo para el proyecto CIAA.

\end{itemize}

En este trabajo fue fundamental los conocimientos y habilidades adquiridos en las diferentes asignaturas de la carrera,  destacando: implementación de manejadores de dispositivos, implementación de sistemas operativos, sistemas operativos de tiempo real y testing de software embebido.

%----------------------------------------------------------------------------------------
%	SECTION 2
%----------------------------------------------------------------------------------------
\section{Próximos pasos}

A continuación, se indican las principales líneas de trabajo futuro para continuar con el desarrollo de la plataforma de emulación.

\begin{itemize}

\item Incorporar otras plataformas de hardware del proyecto CIAA.

\item Emular nuevos periféricos, tales como servo motores, PWM, I2C, etc.

\item Agregar características gráficas entre las conexiones de la placa y los periféricos.

\item Implementar herramientas de depuración que permitan observar los valores de las variables, monitorear el flujo del programa y detectar posibles errores en el código..

\item Implementar otros componentes o funcionalidades proporcionados por las bibliotecas de FreeRTOS, tales como queues, prioridades y hooks.


\end{itemize}